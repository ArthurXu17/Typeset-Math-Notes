\documentclass{article}
\usepackage{commands}
\usepackage[class= MAT 327, doctype= The\ Big\ List]{notes_template}

\begin{document}
\tableofcontents
\newpage
\section{Introduction to Topology}
\begin{enumerate}
    \setlength{\parskip}{1em}
    \item Fix $a < b \in \bR$. Show explicitly that the interval $(a, b)$ is open in $\bru$. Show explicitly that the interval $[a, b)$ is not open in $\bru$.
    
    Let $x\in (a,b)$. Take $\epsilon = \min(x-a, b-x)$. Let $y\in B_{\epsilon}(x)$. If $y = x$, then $y \in (a,b)$. If $y < x < b$, we have that $x - y  = \lvert x -y \rvert < \epsilon \implies x-y < x-a \implies y > a$. Hence $y\in (a,b)$ in this case. If $a < x < y$, we have that $y - x  = \lvert x -y \rvert < \epsilon \implies y-x > b-x \implies y < b$. Hence $y \in (a,b)$ in this case. Thus, $B_{\epsilon}(x)\subseteq (a,b)$ proving that $(a,b)$ is open in $\bru$. $\Box$

    To show that $[a,b)$ is not open, consider an arbitrary open ball centred at $a\in [a,b)$. Note that for all $\epsilon > 0$, $a - \frac{\epsilon}{2}\in B_{\epsilon}(a)$ but $a - \frac{\epsilon}{2} \not\in [a,b)$. Hence every open ball centered at $a$ is not contained in $[a,b)$ which means that $[a,b)$ is not open in $\bru$. $\Box$
    \item Let $X$ be a set and let $\cb = \{ \{x\} : x \in X \}$. Show that the only topology on $X$ that contains $\cb$ as a subset is the discrete topology.
    
    Let $\ct$ be an arbitrary topology such that $\cb\subseteq \ct$. I claim that $\ct = \mathcal{P}(X)$, which by definition is the discrete topology. The forward direction follows by the definition of a topology since a topology is a collection of subsets. For the converse direction, let $A\subseteq X$. If $A = \emptyset$, $A\in \ct$ since we know that $\ct$ is a topology. Otherwise, we can then write $A = \bigcap_{a\in A}\{a\}$ and hence $A$ is a union of elements of $\cb$ and since $\cb\subseteq \ct$, by the definition of a topology, $A\in \ct$. Hence $\ct = \cp(X)$ as desored. $\Box$

    \item Fix a set $X$, and let $\ctf$ and $\ctc$ be the co-finite and co-countable topologies on $X$, respectively.
    \begin{enumerate}
        \setlength{\parskip}{1em}
        \item Show explicitly that $\ctf$ and $\ctc$ are both topologies on X
        
        For co-finite, we first note that $\emptyset\in \ctf$ by definition and $X\in \ctf$ since $X\setminus X = \emptyset$ which is finite. Let $U_{1},U_{2}\in\ctf$. If either are empty, the intersection is empty and thus remains in $\ctf$. Otherwise, say $U_{1} = X\setminus\{x_{1},\dots, x_{m}\}$ and $U_{2} = X\setminus \{y_{1},\dots, y_{n}\}$. Then $U_{1}\cap U_{2} = X\setminus (\{x_{1},\dots, x_{m}\}\cup \{y_{1},\dots, y_{n}\})$ and hence is still in $\ctf$. Now consider a collection $\{U_{\lambda}\}_{\lambda\in \Lambda}$ of open sets in $\ctf$. We can assume they are all non-empty since this will not change the value of the union. So each $U_{\lambda}$ is of the form $X\setminus V_{\lambda}$ where $V_{\lambda}$ is finite. We have that:
        \begin{align*}
            \bigcup_{\lambda\in\Lambda}U_{\lambda} &= \bigcup_{\lambda\in\Lambda}X\setminus V_{\lambda}\\
            &= X\setminus\left(\bigcap_{\lambda\in \Lambda} V_{\lambda}\right)
        \end{align*}
        Note that the arbitrary intersection of finite sets remains finite (this is seen because the arbitrary intersection is a subset of any one of those finite sets) and hence, $\bigcup_{\lambda\in\Lambda}U_{\lambda}$ remains open. Hence the cofinite topology is indeed a topology. $\Box$

        For the cocountable topology, the proof is the same. Since the union of 2 countable sets is still countable, finite intersection of open sets in the cocountable topology is open. Since arbitrary intersections of coutnable sets is countable, arbitrary unions remain open. Hence the cocountable topology is a topology. $\Box$
        \item Show that $\ctf \subseteq \ctc$.
        
        This just follows from the fact that all finite sets are countable. So if $X\setminus U$ is finite (i.e $U\in \ctf$), then $X\setminus U$ is countable and thus $U\in \ctc$. $\Box$
        \item Under what circumstances does $\ctf = \ctc$?
        
        This occurs if and only if $X$ is finite. For the converse direction, we see that $\ctf = \ctc = \cp(X)$ because every subset of $X$ is finite and countable. For the forward direction, consider the contrapositive. Let $X$ be infinite and thus there exists a countable infinite (denumerable) subset $C$ in $X$. Note that $X\setminus C$ is thus an element of the cocountable topology but not the cofinite topology. $\Box$

        \item Under what circumstances does $\ctd = \ctc$?
        
        This occurs if and only if $X$ is countable. For the converse direction, every subset is countable, and thus every subset of $X$ has its complement countable, thus $\ctc = \cp(X) = \ctd$. For the forward direction, consider the contrapositive. Suppose $X$ is uncountable and consider any $x\in X$. Note that $X\setminus\{x\}$ remains uncountable, and so $\{x\} \in \ctd$ but $\{x\}\not\in \ctc$ and hence $\ctc \neq \ctd$ as desired. $\Box$ 
    \end{enumerate}
    \item Let $(X, \ctc)$ be an infinite set with the co-countable topology. Show that $\ctc$ is closed under countable intersections. Give an example to show that it need not be closed under arbitrary intersections.
    
    Let $\{U_{n}\}_{n\in \bN}$ be countably many sets in the co-countable topology. We can assume they are all non-empty since this will not change the value of the union. Thus they are of the form $U_{n} = X\setminus V_{n}$ where $V_{n}$ is countable for all $n\in \bN$. Then, we simply note that:
    \begin{align*}
        \bigcap_{n\in \bN}U_{n} &= \bigcap_{n\in\bN} X\setminus V_{n}\\
        &= X\setminus\left(\bigcup_{n\in\bN} V_{n}\right)
    \end{align*} 
    Note that the countable union of countable sets remain countable so the countable intersection above is still in the cocountable topology. $\Box$

    To show that arbitrary intersections need not remain in the cocountable topology, consider $\bR$ equipped with the cocountable topology. For $x \in \bR$ with $x\geq 0$, define $U_{x} = \bR \setminus \{x\}$ which is an element of the cocountable topology. Then consider the arbitrary intersection of these sets. We see that $\bigcap_{x\geq 0} U_{x} = \bR_{< 0}$. Note that $\bR_{< 0}$ is not an element of the cocountable topology since it's complement is the non negative real numbers which is not countable.

    \item Let $X$ be a nonempty set, and fix an element $p \in X$. Recall that
    \begin{align*}
        \ct_{p} &= \{U\subseteq X:\ p\in U\}\cap\{\emptyset\}
    \end{align*}
    is called the point topology at $p$ on X. Show that $\ct_{p}$ is a topology on $X$.

    We have $p\in X$ so $X\in \ct_{p}$ and that $\emptyset \in \ct_{p}$. For abitrary unions, if $p\in U_{\lambda}$ for all $\lambda \in \Lambda$, then $p\in \bigcup U_{\lambda}$. Similarly, if $p\in U_{1}$ and $p\in U_{2}$, then $p\in U_{1}\cap U_{2}$. Hence this is a topology. $\Box$
    \item Define the ray topology on $\bR$ as:
    \begin{align*}
        \ctr &= \{(a,\infty): a\in \bR\}\cup \{\emptyset, \bR\}
    \end{align*}
    Show that $\ctr$ is a topology on $\bR$. Be sure to think carefully about unions.

    By definition, $\emptyset, \bR \in \ctr$. For intersections, if the empty set is contained in the intersection, the intersection is the empty set. If $\bR$ is contained in the intersection of 2 open sets, the other set (which is open) will be the result and thus the intersection is open. Otherwise, consider $(a,\infty) \cap (b,\infty)$. Simply note that this is equal to $(\max(a,b), \infty)$ because $x > a$ and $x > b$ if and only if $x > \max(a,b)$. For unions, if $\bR$ is contained in the union the result is $\bR$ which is open. We can assume there are no empty sets in the union as that will not change the result. Now consider an arbitrary collection of intervals $(a_{\lambda}, \infty)$, and let $S = \{a_{\lambda}: \lambda\in \Lambda\}$. We consider 2 cases, $S$ is bounded below and $S$ is not bounded below. 
    
    In the first case, by the completeness of $\bR$, we know that $\inf S$ exists, and I claim that $\bigcup (a_{\lambda}, \infty) =  (\inf S, \infty)$. For the forward inclusion, if $x \in (a_{\lambda_{0}}, \infty)$ for some $a_{\lambda_{0}}\in S$, we have that $x >  a_{\lambda_{0}} \geq \inf S$ and hence $x\in (\inf S, \infty)$. For the converse inclusion, if $x > \inf S$, by the definition of the greatest lower bound, there exists an $\inf S < a_{\lambda_{0}} < x$ (otherwise $x$ would be a greater lower bound). Hence $x \in \bigcup (a_{\lambda}, \infty)$. 

    Now suppose that $S$ is not bounded below. I claim then that $\bigcup (a_{\lambda}, \infty) = \bR$. The forward direction follows since each interval is a subset of $\bR$. For the converse direction, given $x\in \bR$, since $S$ is not bounded below, there exists $a_{\lambda_{0}} \in S$ with $a_{\lambda_{0}} < x$ and thus $x\in \bigcup (a_{\lambda}, \infty)$ as desired.

    Hence $\ctr$ is a topology. $\Box$
    \item Let $(X, \ct )$ be a topological space, and let $A \subseteq X$ be a set with the property that for every $x \in  A$, there is an open set $U_{x} \in \ct$ such that $x \in U_{x} \subseteq A$. Show that $A$ is open.
    
    This simply follows by noting that $A = \bigcup_{x\in A} U_{x}$. Converse direction follows since each $U_{x}$ is a subset of $A$ forward direction follows because for all $x\in X$, $x\in U_{x} \subseteq \bigcup_{x\in A} U_{x}$. Since each $U_{x}$ is open, $A$ is open. $\Box$
    \item Let $(X, \ct )$ be a topological space, and let $f : X \to Y$ be an injective (but not necessarily surjective) function. Is $\ct_{f} := \{ f(U) : U \in T \}$ necessarily a topology on $Y$ ? Is it
    necessarily a topology on the range of $f$?

    This is not necessarily a topology on $Y$. Let $Y = \bR\times\{0,1\}$ and let $X$ be $\bru$. Define the function $f: X\to Y$ by $x\mapsto (x,0)$. We see that $f(U) = U\times \{0\}$ for all non empty sets $U\in \bR$ and $f(\emptyset) = \emptyset$. In particular, $Y\not\in \ct_{f}$ and so this is not a topology on $Y$. $\Box$

    It is a topology on the range of $f$. We have that $f(\emptyset) = \emptyset$ and $f(X) = \range f$ (and $\emptyset$ and $X$ are open in $X$). Note that since $f$ is injective, we have that $f(U_{1}\cap U_{2}) = f(U_{1})\cap f(U_{2})$. The forward inclusion is true for all functions since if $y = f(x)$ with $x\in U_{1}\cap U_{2}$, then $y\in f(U_{1})$ and $y\in f(U_{2})$. For the reverse inclusion, say $y = f(x_{1})$ and $y = f(x_{2})$ with $x_{i}\in U_{i}$. Since $f$ is injective, we have that $x_{1} = x_{2}$ and hence they are both elements of $U_{1}$ and $U_{2}$. Hence $y\in f(U_{1}\cap U_{2})$. Since  $f(U_{1}\cap U_{2}) = f(U_{1})\cap f(U_{2})$, we note that $\ct_{f}$ is closed under finite intersections as $U_{1}\cap U_{2}$ will be open in $X$ whenever $U_{1}$ and $U_{2}$ are both open in $X$.

    Now let's consider arbitrary unions. We first prove that for arbitrary collection of sets $\{U_{\lambda}\}_{\lambda\in\Lambda}$, $f\left(\bigcup_{\lambda\in\Lambda}U_{\lambda}\right) = \bigcup_{\lambda\in \Lambda}f(U_{\lambda})$. The forward direction follows because if $y = f(x)$ where $x\in \bigcup_{\lambda\in \Lambda}U_{\lambda}$, then $x\in U_{\lambda_{0}}$ and so $y =f(x)\in f(U_{\lambda_{0}})\subseteq \bigcup_{\lambda\in \Lambda}f(U_{\lambda})$. For the converse direction, if $y \in f(U_{\lambda_{0}})$, then $y = f(x)$ where $x\in U_{\lambda_{0}}\subseteq \bigcup_{\lambda\in \Lambda}U_{\lambda}$ and hence $y= f(x)\in f\left(\bigcup_{\lambda\in\Lambda}U_{\lambda}\right)$. Hence $\ct_{f}$ is closed under arbitrary union since $\bigcup_{\lambda\in\Lambda}U_{\lambda}$ will be open in $X$ whenever each $U_{\lambda}$ is open in $X$.
    
    Hence $\ctf$ is indeed a topology on the range of $f$. $\Box$

    \item Let $X$ be a set and $\ct_{1}$ and $\ct_{2}$ be two topologies on $X$. Is $\ct_{1}\cup \ct_{2}$ a topology on $X$? Is $\ct_{1}\cap \ct_{2}$ a topology on $X$? If yes, prove it. If not, give a counterexample.
    
    Unions of topologies are not necessarily topologies. Consider $X = \{1, 2, 3\}$ and $\ct_{1} = \{\emptyset, \{1\}, X\}$ and $\ct_{2} = \{\emptyset, \{2\}, X\}$. Note that $\ct_{1} \cup \ct_{2} = \{\emptyset, \{1\}, \{2\}, X\}$. This is not a topology since $\{1,2\} = \{1\}\cup \{2\} \not\in\ct_{1} \cup \ct_{2}  $ but both of $\{1\}$ and $\{2\}$ are in $\ct_{1} \cup \ct_{2}$. 

    Intersections of topologies are topologies. Let $X$ be a set and $\ct_{1}$ and $\ct_{2}$ be arbitrary topologies. By definition, $\emptyset, X\in \ct_{1}\cap \ct_{2}$. Let $U,V\in \ct_{1}\cap \ct_{2}$. By the definition of a topology, $U\cap V \in \ct_{1}$ and $U\cap V \in \ct_{2}$. Hence, $U\cap V\in \ct_{1}\cap \ct_{2}$. If $\{U_{\lambda}\}_{\lambda\in \Lambda}$ are all open sets in $\ct_{1}\cap\ct_{2}$, then by the definition of a topology $\bigcup U_{\lambda}\in \ct_{1}$ and $\bigcup U_{\lambda}\in \ct_{2}$. Hence, $\bigcup U_{\lambda}\in \ct_{1}\cap\ct_{2}$ and we have that $\ct_{1}\cap \ct_{2}$ is still a topology. $\Box$
    \item Let $X$ be an infinite set. Show that there are infinitely many distinct topologies on $X$
    
    Note that for any $x\in X$, $\ct_{x} = \{\emptyset, \{x\}, X\}$ forms a topology on $X$. We see that $\emptyset, X\in \ct_{x}$ and that arbitrary unions and intersections stay in $\ct_{x}$. Thus to get infinitely many distinct topologies, we just consider $\{\ct_{x}: x\in X\}$. $\Box$
    \item Fix a set $X$, and let $\phi$ be a property that subsets $A$ of $X$ can have. For example, $\phi$ could be “$A$ is countable”, or “$A$ is finite”. $\phi$ could be “$A$ contains $p$” or “$A$ doesn't contain $p$” for a fixed point $p \in X$. If $X = \bR$, $\phi$ could be “$A$ is an interval” or “$A$ contains uncountably many irrational numbers less than $\pi$”. Define
    \begin{align*}
        T_{\text{co-}\phi} = \{U \subseteq X : U = \emptyset, \text{ or } X \setminus U\text{ has }\phi\}
    \end{align*}
    Under what assumptions on $\phi$ is $T_{\text{co-}\phi} $ a topology on $X$? Which topologies we have seen so far can be described in this way, using which $\phi$?

    First we note that $U\neq\emptyset\in T_{\text{co-}\phi}$ if and only if $U = X\setminus V$ where $V$ satisfies $\phi$ (in particular $V = X\setminus U$). Finite intersections $(X\setminus V_{1})\cap (X\setminus V_{2})$ are of the form $X\setminus (V_{1}\cup V_{2})$, so we require that finite unions of sets satisfying $\phi$ still satisfy $\phi$. Also $\bigcup X\setminus V_{\lambda} = X\setminus \bigcap V_{\lambda}$, so we require arbitrary intersections of sets satisfying $\phi$ to still satisfy $\phi$. Finally since we want $X\in T_{\text{co-}\phi} $, we require that $\emptyset$ satisfies $\phi$. Cocountable, cofintite, and particular point topology all satisfy this where the properties are ``is finite'', ``is countable'' and ``does not contain $p$''. We can also phrase the discrete topology in this way with the property ``is a set''. We can phrase the trivial topology in this way by using the property ``is the empty set''. The ray topology can be phrased as ``is an interval of the form $(-\infty, a)$ or is equal to the emptyset''.

    \item Let $\{\ct_{\alpha}:\alpha\in I\}$ be a collection of topologies on a set $X$, where $I$ is some indexing set. Prove that there is a unique finest topology that is refined by all the $T_{\alpha}$. That is, prove that there is a topology $\ct$ on $X$ such That
    \begin{enumerate}
        \item $\ct_{\alpha}$ refines $T$ for every $\alpha \in I$.
        \item If $T'$ is another topology that is refined by $T_{\alpha}$ for every $\alpha\in I$, then $T$ is finer than $T'$
    \end{enumerate}
    We simply take $\ct$ = $\bigcap_{\alpha\in I} \ct_{\alpha}$. We see that this is a subset of each $\ct_{\alpha}$ which means that condition a) is satisfied. Also if $T'$ is a subset of each $\ct_{\alpha}$ it is thus a subset of their intersection which satisfies condition b). It remains to be shown that this is actually still a topology.

    We see that $\emptyset, X$ are elements of each $\ct_{\alpha}$ and is thus an element of $\ct$. If $U_{1},U_{2}\in \ct$, then $U_{1},U_{2}\in \ct_{\alpha}$ for all $\alpha\in I$, so $U_{1}\cap U_{2}\in \ct_{\alpha}$ for all $\alpha\in I$ and hence $U_{1}\cap U_{2}\in \ct$. For unions, suppose $\{U_{\lambda}\}_{\lambda\in \Lambda}$ is a collection of sets in $\ct$. They are thus contained in each $\ct_{\alpha}$. Since each $\ct_{\alpha}$ is a topology, $\bigcup_{\lambda\in\Lambda}\in \ct_{\alpha}$ for all $\alpha$ and hence the union is contained in $\ct$ as desired. $\Box$
    \item This extends Exercise 1.8. Show that $f$ being injective is necessary. That is given an example of a topological space $(X,\ct)$ and a non-injective function $f:X\to Y$ such that $\ct_{f}$ is a topology on the range of $f$ and an example where it is not a topology.
    
    For an example where $\ct_{f}$ is a topology consider $f: \bru\to \bru$ where $f$ is the constant map that maps everything to 1. $\ct_{f}$ in this case is simply $\{\emptyset, \{1\}\}$ which is just the trivial topology on the range of $f$ which is $\{1\}$.

    For an example where $\ct_{f}$ is not a topology, let $X = \{0,1,2,3\}$ and $Y = \{0,1,2\}$. Define $f:X\to Y$ where $f(0) = f(1) = 0$, $f(2) = 1$ and $f(3) = 2$ (note that $f$ is surjective here). Define the following topology on $X$: $\ct = \{\emptyset, \{0,2\}, \{1,3\}, X\}$. The resulting set for $\ct_{f}$ is $\{\emptyset, \{0,1\}, \{0,2\}, Y\}$. Note however that this is not a topology since $\{0,1\}\cap \{0,2\} = \{0\}$ but $\{0\}\not\in \ct_{f}$.

    \item Working in $\bru$:
    \begin{enumerate}
        \setlength{\parskip}{1em}
        \item Show that every nonempty open set contains a rational number
        
        Let $U$ be a nonempty open set. Let $x\in U$. Since $U$ is open, there exists an $\epsilon$ greater than 0 such that the interval $(x-\epsilon, x+\epsilon)\subseteq U$. By the density of the rationals, there exists a rational number $q$ contained in the interval which is a subset of $U$ as desired. $\Box$

        \item Show that there is no uncountable collection of pairwise disjoint open subsets of $\bR$
        
        Suppose for contradiction an uncountable collection of pairwise disjoint open subsets of $\bR$ exists. By part a), each of these open subsets would contain a rational number and since the subsets are all disjoint, each of these rational numbers would be distinct and we would end up with an uncountable collection of rational numbers. This is a contradiction since there are countably many rational numbers. $\Box$
    \end{enumerate}
\end{enumerate}
\end{document}